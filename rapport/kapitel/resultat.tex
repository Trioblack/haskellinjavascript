\section{Resultat}
todo

\subsection{HIJi} 
HIJi erbjuder användaren ett GHCi-liknande gränssnitt. HIJi fungerar som en fasad in i programmet. 
HIJi tar input genom att användaren skriver in Haskellkod som därefter tolkas av parsern och slutligen evalueras uttrycket. Resultatet av uttrycket visas på raden under.

HIJi är skapat för att likna GHCi i så stor utsträckning som möjligt. Forskning in
%TODO get source from cp-book at home

HIJi är tänkt att erbjuda användaren liknande möjligheter som GHCi. 

Fördelen med HIJi framför GHCi är att användaren ej behöver ladda ner den tunga GHC-kompilatorn på sin personliga dator för att testa enkla Haskelluttryck direkt i webbläsaren.

Nackdelar gentemot GHCI är att HIJi är en nedbandat version utav GHCi. HIJi kan bara evaluera enklare uttryck. Det finns i dagsläget inga möjligheter att ladda upp hela Haskell-filer för att köra dem. Att som i GHCi på ett enkelt sätt kolla upp vilka typer en funktion stöds ej.

HIJi använder sig utav JQuery för att få ett unisont stöd på alla moderna webbläsare.


\subsection{Parser} 
%TODO

\subsection{Abstrakt syntaxträd} 
%TODO

\subsection{Typcheckare} 
%TODO
