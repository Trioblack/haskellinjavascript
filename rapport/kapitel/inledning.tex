\section{Inledning}
På vissa av Chalmers och Göteborgs Universitets datorvetenskapliga program är den första programmeringskursen i Haskell \citep{haskell98} och för en del av de nya eleverna är inlärningströskeln hög. De studenter som börjar på de datavetenskapliga programmen på Chalmers och Göteborgs Universitet är allt från nybörjare till mycket kompetenta inom programmering. De flesta saknar dock kunskaper kring funktionell programmering. Skillnaden mellan ett funktionellt och ett objektorienterat programmeringsspråk är stora och omställningen hur programmeringsrelaterade problem behöver angripas  är inte enkelt för de flesta nybörjare. Vi tror att ett interaktivt webbverktyg skulle kunna sänka tröskeln och underlätta undervisningen. Ett webbverktyg medför även att en kompilator som \emph{Glasgow Haskell Compiler} (GHC) \citep{ghc} ej behöver installeras. Webbens stöd för interaktivitet gör det möjligt att snabbt visa funktionsdeklarationerna för de inbyggda funktionerna och att enkelt evaluera funktionerna och testa sig fram till olika resultat.

Många programmerare kommer inte i kontakt med funktionell programmering  och med hjälp av ett interaktivt webbverktyg som är enkelt för användaren att använda är vår förhoppning att fler programmerare och studenter ska komma i kontakt med funktionell programmering, och i synnerhet Haskell. Då flera moderna objektorienterade programmeringsspråk börjar ta funktionalitet och begrepp från funktionella programmeringsspråk så är det extra viktigt att programmerare kommer i kontakt med funktionell programmering. Ett exempel på detta är \emph{C\#} som i senare versioner har fått stöd för bland annat lambda-funktioner (anonyma funktioner) \citep{csharp}. 

En fördel med att ha tolken på webben är att det enda som behövs för att använda den är en javascriptkompatibel webbläsare, något som följer med i alla moderna operativsystem. Detta betyder att de användare som befinner sig inom vår målgrupp redan har den programvaran som behövs på sina hemdatorer för att använda sig av vårt program.  

% TODO  detta e out of context`
Det som gör Haskell speciellt är att det är ett starkt statiskt typcheckat och funktionellt programmeringsspråk med lat evaluering. % TODO: Citera, skriv om ifall det är direktöversatt
% jag bytte från strikt semantik till lat evaluering:p
Att språket är funktionellt innebär bland annat att funktioner är \emph{first-class citizens} och kan därmed användas som parametrar och returneras från andra funktioner precis som vilken annan typ som helst.

Lat evaluering innebär att evalueringen av ett uttryck inte kommer utföras förrän resultatet av uttrycket behövs. Om uttrycket inte behövs  kommer interpretatorn att ignorera det. 
Lat evaluering gör att programmeraren inte behöver bry sig om exekveringsordningen av ett program. Detta ger prestandaförbättringar eftersom ett uttryck inte evalueras alls om det inte behövs \citep{hudak89}.
Lat evaluering gör det också möjligt att använda sig av oändliga datastrukturer, till exempel oändliga listor. Språket blir därmed mer uttrycksfullt. 

John Hughes argumenterar för att  funktionella språk som stödjer lat evaluering erbjuder större möjligheter än imperativa språk att skriva modulära program. Detta för att funktionella språk som Haskell stödjer \emph{higher order functions}, en funktion som tar en annan funktion som argument, och lat evaluering vilket är tekniker som kan användas för att binda samman olika moduler.
Dessa två programspråksegenskaper bidrar till att program skrivna i Haskell generellt sett är kortare och går fortare att skriva än motsvarande program skrivet i ett imperativt programmeringsspråk \citep{why}.

Med ovan nämnda resonemang ser vi det som ovärderligt för programmerare att komma i kontakt och lära sig funktionell programmering. 
Förhoppningen är att vår haskelltolk i Javascript ska kunna användas som grund för att i framtiden göra en interaktiv läroplattform för nybörjare i funktionell programmering. 


\subsection{Syfte}
Syftet är en implementera en fungerande haskelltolk i Javascript. Den ska kunna tolka en delmängd av haskell-specifikationen så den kan användas för att göra exempelvis interaktiva tutorials för nybörjare.
Meningen är att dessa ska kunna köras i en vanlig webbläsare utan att ladda ner en haskellkompilator, till exempel GHC, eller behöva lära sig svårbegripliga kommandon.

\subsection{Metod}
Projektet bestod av att planera, designa och implementera haskelltolken. Vi arbetade efter en iterativ modell där nya funktioner lades till undan för undan. Detta fungerade bra eftersom vi tidigt fick en fungerande prototyp att utgå från. Arbetet delades upp i separata moduler som utvecklades relativt frånskilt från varandra för att undan för undan fasas ihop till det slutgiltiga resultatet. 

%I vårt arbete har vi implementerat parser, typcheckare och interpretator parallellt med varandra och utökar de olika modulernas funktionalitet iterativt.
%Vi hade tänkt följa den här planen genom varje milstolpe genom att utöka parsern, typcheckaren och interpretatorn med ny funktionalitet.

%Ett första delmål är att göra en enkel implementation av lambda calculus då mer avancerade funktionella programspråksegenskaper kan implementeras som detta \citep{jones87}.
% Parsern implementeras med hjälp av ett parser combinator bibliotek kallat JSParse \citep{jsparse}. Detta ger oss möjlighet att att på ett enkelt sätt implementera både den kontextfria och icke kontextfria delen av Haskell. Parsern skapar ett syntaxträd som skickas vidare till typcheckaren. I typcheckaren dekoreras syntaxträdet med typinformation innan det slutligen skickas till interpretatorn.

%Vi använder även JSParse, en parser combinator, för att bygga ett syntaxträd som skickas till typcheckaren och interpretatorn. I typcheckaren dekoreras syntaxträdet med typinformation.

%En interaktiv kommandotolk som kan köras i en webbläsare kommer att utvecklas. Den ska ge användaren möjlighet att skriva Haskell-funktioner och exekvera dem på ett liknande sätt som i GHCi. 
%Vi kommer att integrera jQuery \citep{jquery} för att få ett unisont stöd för samtliga webbläsare. jQuery kommer även underlätta arbetet med att skapa ett enkelt och stilrent interaktivt gränssnitt.

\subsection{Avgränsningar} 
Att tolka Haskell i Javascript är inget trivialt projekt och därför valde vi att avgränsa implementationen till de mest centrala och viktiga delarna i haskellspecifikationen. De delar vi valde att fokusera på är:
        \begin{itemize}
            \item{Lambda-funktioner, namngivna funktioner}
            \item{Typer, generella typer, algebraiska datatyper}
            \item{Typklasser}
            \item{Pattern matching}
            \item{Guards}
        \end{itemize}
Med dessa delar implementerade kan de flesta enklare haskellprogram köras och bör vara tillräckligt för de flesta nybörjare. Vi beslutade att ej lägga någon större tid på att skapa en användarvänlig webbsida, utan fokus låg på att skapa haskelltolken. Dock valde vi att utveckla en kommandotolk som skulle köras på en webbsida för att kunna kommunicera med haskelltolken. 
Slutligen beslutades det att inte optimera haskelltolken utan målet var att göra en fungerande implementation. Detta för att en nybörjare sällan behöver eller kommer skriva program som behöver hög prestanda.

