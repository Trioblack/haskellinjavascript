\documentclass[a4paper, 12pt]{article}

\usepackage{float}
\usepackage[T1]{fontenc}
\usepackage[utf8]{inputenc}
\usepackage[swedish]{babel}
\usepackage{graphicx}
\usepackage{natbib} % ger Harvard-referenser
\usepackage{graphicx}
\usepackage{listings}
\usepackage{color}

\setlength{\parskip}{12pt}
\setlength{\parindent}{0pt}


\hyphenpenalty=10000
\tolerance=5000

\begin{document}

\title{Opponering på Databases for Dummies 2}
\author{
    Adam Bengtsson, 
    Mikael Bung, 
    Johan Gustafsson, 
    Mattis Jeppsson 
}
\date{\today}
\maketitle
    

\emph{Databases for dummies 2} är en rapport som beskriver ett program som ska användas för att administrera databaser, utan att som användare behöva några större förkunskaper. Vi har läst rapporten i opponeringssyfte och har en del tankar kring dess innehåll.

Överlag tycker vi rapporten var intressant och väldigt väl beskriver arbetsgången av projektet. Vi är positiva till att alla användarfall med högsta prioritet framgångsrikt har implementerats. Programmets syfte, att underlätta databashantering, ser ut att vara uppfyllt. Det är dock en del stycken i rapporten som vi skulle vilja få mer utförligt förklarade.

Från rapporten har vi fått uppfattningen att en hel del förkunskaper om databaser krävs för att kunna använda programmet. Detta får oss att undra om inte alternativa representationer av databaser skulle förenkla för nybörjare. I rapporten beskrivs ej vilka konceptuella modeller som har utforskats för att representera databaser. Till exempel att använda ER-diagram för att på ett grafiskt sätt beskriva en databas för användare, se sammanhang mellan olika databaser och eliminera svårbegripliga begrepp som till exempel \emph{foreign keys}.

Vi är också fundersamma över designbeslutet att bara använda sig av en View-klass. I rapporten, sidan 7, rad 9,  nämns det att en namnkonvention för att kunna navigera i View-filen var nödvändig då den växt ur proptotion och därmed blivit svår att navigera i. Som motivering till beslutet att inte dela upp View-klassen i flera mindre klasser, nämner rapporten att det vore alltför tidskrävande utan att närmare gå in på orsakerna till detta.

Det grafiska användargränssnittet som beskrivs har ett modernt och profesionellt utseende, samtidigt som det ser tillräckligt enkelt ut för att användas av nybörjare med tillräckliga förkunskaper. Programmet ser självförklarande ut och liknar andra Windowsprogram i menyer vilket är en fördel när det riktar sig till nybörjare. 

Vidare beskrivs i kapitel 5. Discussion, sidan 22, rad 14, att till en början användes en enda controller-klass. Den hade prestandaproblem på grund av många jämförelser i sin action-listener vilket skapade fördröjningar på flera sekunder. I rapporten hävdas att prestandeproblemen upphörde när controller-klassen delades upp i flera mindre klasser. Vi är fundersamma över hur dessa prestandaproblem var möjliga och skulle gärna se en mer utförlig förklaring kring problemet. 

Vi har även synpunkter på att det i teorikapitlet, sidan 2, finns ett avsnitt som beskriver vad ER-diagram är, utan att rapporten som följer använder dessa. Detta medför att det nämnda avsnittet blir överflödigt för förståelsen av rapporten.

Slutligen vill vi säga att rapporten har en bra struktur och beskriver programmet som utvecklades på ett fullt tillräckligt vis. 
\end{document}
